\documentclass[a4paper, 12pt, oneside]{scrbook}

\input{settings.tex}

\bibliography{bibliography.bib}
\begin{document}

\frontmatter

% Hierin müssen Matrikelnummer Name usw. gesetzt werden.
\def\doctype{Dokumententyp}
\def\title{Entwicklung einer Webanwendung unter Verwendung des Angular-Frameworks,
welche die Verwaltung von Schichtwechseln für Mitarbeiter unterstützt}
\def\author{Nora Leuchner}

\begin{titlepage}

	\vspace{9mm}

	\begin{center}
		\vspace{5mm}

		\huge \title

		\vspace{13.2pt}

		%\large \doctype

		\vspace{42.6pt}

		\large Studienarbeit T3\_3101

		\vspace{42.6pt}

		\small des Studienganges Angewandte Informatik an der \\
		\large Dualen Hochschule Baden-Württemberg Mosbach

		\vspace{14.2pt}

		\includegraphics[height=1.5cm]{images/logo-dhbw.pdf}

		\vspace{42.6pt}

		\small von \\
		\large \author
	\end{center}

	\vspace{110pt}

	\begin{table}[h]
		\centering
		\begin{tabular}{ll}
			% \small Bearbeitungszeitraum            & XXX Wochen     \\
			\small Matrikelnummer, Kurs            & 4753795, INF21B \\
			\small Gutachter der Dualen Hochschule & Philipp Abele   \\
		\end{tabular}
	\end{table}

	\vspace{49.7pt}

	\fancypagestyle{empty}{
		\fancyhf{}
		\fancyfoot[C]{\today}
	}

\end{titlepage}

\pagenumbering{gobble}
\input{prefix/eigenstaendigkeit.tex}

% Kurzfassung
\section*{Kurzfassung}
Diese Studienarbeit behandelt die Entwicklung einer Anwendung zum Verwalten von Schichtwechseln unter der Verwendung des Google Design Sprints. Zunächst wurde die UX entwickelt und anschließend darauf basierend ein Benutzerkonzept, um die Nutzerfreundlichkeit und Effektivität der Anwendung zu gewährleisten. Im Anschluss daran wurde eine Nutzerumfrage durchgeführt, um Feedback von den Anwendern zu sammeln. Diese Umfrage zielte darauf ab, herauszufinden, wie das Design bei den Nutzern ankommt, ob die Anwendung intuitiv ist und welche Verbesserungsvorschläge oder zusätzlichen Informationen von den Nutzern gewünscht werden. Die Ergebnisse der Umfrage wurden sorgfältig ausgewertet. Es stellte sich heraus, dass die Nutzer das Design im Allgemeinen gut fanden, jedoch einzeln Verbesserungsvorschläge und der Wunsch nach zusätzlichen Informationen geäußert wurde. Basierend auf diesen Erkenntnissen wurde das Benutzerkonzept überarbeitet, um besser auf die Bedürfnisse und Wünsche der Nutzer einzugehen. In der letzten Phase der Entwicklung wurde anhand des überarbeiteten Benutzerkonzepts ein Prototyp mit Angular erstellt. Diese dient als Grundlage für die weitere Entwicklung und Verfeinerung der Anwendung, um eine optimale Nutzererfahrung zu gewährleisten.

\clearpage

% Abstract
\section*{Abstract}

This student research project deals with the development of an application for managing shift changes using the Google Design Sprint. First, a UX and user concept was developed to ensure the user-friendliness and effectiveness of the application. Following this, a user survey was conducted to gather feedback from users. This survey aimed to find out how the design was received by users, whether the application was intuitive and what suggestions for improvement or additional information users would like to see. The results of the survey were carefully analyzed. It turned out that the users generally liked the design, but that there were individual suggestions for improvement and a request for additional information. Based on these results, the user concept was modified to better meet the needs and wishes of the users. In the final phase of development, a prototype was created with Angular based on the updated user concept. This serves as the basis for further development and refinement of the application to ensure an optimal user experience.

\tableofcontents

% Abbildungsverzeichnis
\cleardoublepage
\phantomsection
\addcontentsline{toc}{chapter}{\listfigurename}
\pagenumbering{Roman}
\listoffigures

% Tabellenverzeichnis
% \cleardoublepage
% \phantomsection
% \addcontentsline{toc}{chapter}{\listtablename}
% \listoftables

%Codebeispiel Verzeichnis
% \cleardoublepage
% \phantomsection
% \addcontentsline{toc}{chapter}{\lstlistlistingname}
% \lstlistoflistings

% Abkürzungsverzeichnis (siehe Ordner "content")
\chapter{Abkürzungsverzeichnis}
\begin{acronym}
    \acro{GDS}{Google Design Sprint}
\end{acronym}

\mainmatter

%%%%%% Inhalt (siehe Ordner "content") %%%%%%
\chapter{Einleitung}

In der modernen Arbeitswelt spielen flexible Arbeitszeitmodelle eine zunehmend wichtiger werdende Rolle, um den vielfältigen Anforderungen sowohl der Arbeitgeber als auch der Arbeitnehmer gerecht zu werden. 
Laut einer Studie der International Labour Organization wird Flexibilität in der Arbeitszeitgestaltung als entscheidender Faktor für die Vereinbarkeit von Beruf und Privatleben angesehen und kann die Zufriedenheit und Produktivität der Mitarbeiter signifikant erhöhen \cite[S.12 ff.]{ilo2018}. 
Ein zentrales Element dieser Flexibilität ist die Möglichkeit für Mitarbeiter, Schichten untereinander zu tauschen. 
Oft werden Schichttauschvereinbarungen privat organisiert, beispielsweise über Messenger-Dienste wie WhatsApp. 
%Problemstellung
In vielen Unternehmen organisieren Mitarbeiter ihre Schichtwechsel eigenständig und informell, oft über Messaging-Dienste wie WhatsApp. Dies gilt auch für die Mitarbeiter dieser Firma, die ihre Schichtwechsel in einer WhatsApp-Gruppe privat koordinieren. Ein typisches Beispiel einer solchen Liste für den Monat Januar sieht wie folgt aus:

\begin{figure}[h]
    \centering
    \includegraphics[clip,width=0.25\linewidth]{images/WhatsAppListe.png}
    \caption[Beispiel der Schichtwechsel-Liste aus WhatsApp für Januar]{Beispiel der Schichtwechsel-Liste aus WhatsApp für Januar}
    \label{WhatsAppListe}
\end{figure}

Hierbei steht (S) für „Suche“ und (B) für „Biete“, gefolgt von den jeweiligen Schichtzeiten und den Initialen der Personen.

Diese manuelle Methode bringt mehrere Herausforderungen und Ineffizienzen mit sich. Die fortlaufende Aktualisierung der Liste in WhatsApp führt oft zu einer unübersichtlichen Flut an Informationen, bei der Änderungen oder neue Einträge leicht übersehen werden können. 
Außerdem erfordert die aktuelle Methode von den Mitarbeitern, die gesamte Kommunikation und Liste in WhatsApp ständig zu überwachen, was zeitaufwendig und unpraktisch ist. Diese Herausforderungen beeinträchtigen die Effizienz der Schichtorganisation und führen zu mehr Unzufriedenheit bei den Mitarbeitern.

Das Ziel dieser wissenschaftlichen Arbeit ist es, ein benutzerfreundliches und effizientes UX Design Konzept zur Organisation von Schichtwechseln zu entwickeln. Dieses Konzept soll die aktuellen Probleme der ineffizienten WhatsApp-basierten Methode adressieren und eine strukturierte, leicht zugängliche und übersichtliche Alternative bieten.
\chapter{Grundlagen}

\section{Google Design Sprint}

Der Google Design Sprint \ac{GDS} ist eine von Jake Knapp bei Google Ventures entwickelte Methodik zur schnellen und effizienten Problemlösung sowie Produktentwicklung, insbesondere für Herausforderungen, die sich aus der dynamischen Natur des Marktes und den sich verändernden Produktanforderungen ergeben. 
Ziel dieser Methode ist es, innerhalb eines Zeitraums von fünf Tagen einen Prototyp zu entwickeln und zu evaluieren. 
Diese Methode bietet den Vorteil, dass nicht auf die Markteinführung gewartet werden muss, um Feedback zu erhalten. Stattdessen können dringende Fragen sofort beantwortet werden \cite[S.98 f.]{Design_Sprint}.

Die fünftägige Methode, wie in Abbildung \ref{GDS} dargestellt, verläuft wie folgt:

\begin{figure}[h]
    \centering
    \includegraphics[clip,width=0.75\linewidth]{images/GDS.png}
    \caption[Ablauf eines GDS]{Ablauf eines GDS \cite{GDS_Abbildung}}
    \label{GDS}
\end{figure}

Am ersten Tag geht es darum, das Problem zu verstehen und den Fokus für die Woche festzulegen. Das Team definiert das langfristige Ziel und identifiziert die Herausforderung. 

Am zweiten Tag konzentriert sich das Team auf die Bewältigung bereits bekannter Herausforderungen. Anders als bei herkömmlichen Brainstorming-Sitzungen arbeiten die Teammitglieder einzeln an Lösungsansätzen und folgen einem strukturierten vierstufigen Prozess, um das kritische Denken zu fördern. 

Am dritten Tag trifft das Team Entscheidungen darüber, welche Idee als Prototyp entwickelt und getestet werden sollen. Dabei kommt die fünfstufige "Sticky Decision"-Methode zum Einsatz, um die besten Lösungen zu identifizieren. Anschließend wird ein detaillierter Prozessplan für den Prototypen erstellt. 

Am vierten Tag wird ein realitätsnaher Prototyp entwickelt. Das Ziel ist es, eine testbare Version der Lösung zu erstellen, die am nächsten Tag mit echten Nutzern evaluiert werden kann. 

Der letzte Tag ist für das Testen des Prototyps reserviert. Das Team sammelt Feedback von echten Nutzern und erhält Einsicht, ob die Lösung in der Praxis funktioniert und welche Anpassungen nötig sind. Dieses Feedback ist entscheidend, um die Stärken und Schwächen der entwickelten Lösung zu identifizieren \cite[S.22 ff.]{Design_Sprint}.
\chapter{Entwicklung der User Experience mit Verwendung des Google Design Sprints}

\chapter{Weiterentwicklung des Benutzerkonzepts auf Basis der Ergebnisse der Nutzerumfrage}

Basierend auf den Ergebnissen der Nutzerumfrage wird das Benutzerkonzept weiterentwickelt (siehe Abbildung \ref{Bedienungskonzept_V2}). Mehrere Änderungen sind eingebunden, um die Benutzerfreundlichkeit zu verbessern.

\begin{figure}[h]
    \centering
    \includegraphics[clip,width=1\linewidth]{images/Bedienungskonzept_V2.png}
    \caption[Weiterentwickeltes Benutzerkonzept]{Weiterentwickeltes Benutzerkonzept (Seite zum Registrieren und zum Einloggen sind gleich geblieben und sind deswegen nicht mit abgebildet)}
    \label{Bedienungskonzept_V2}
\end{figure}

Nutzer haben nun in den Einstellungen die Möglichkeit, Tauschangebote in ihren persönlichen Kalender zu exportieren. Dies erleichtert die Planung und Übersicht ihrer Schichtwechsel.

Außerdem werden Nutzer beim Hinzufügen einer neuen Schicht nach zusätzlichen Details gefragt. Insbesondere wird erfragt, ob ein Trainee bei der Schicht anwesend sein wird oder ob die Anfrage dringend ist. Dringende Anfragen sind visuell hervorgehoben, indem sie rot hinterlegt sind. Dies gilt sowohl im normalen als auch im Dark Mode, wie auf den Übersichtsseiten dargestellt.

Wenn Nutzer im Dropdown-Menü nur „später“ oder „früher“ auswählen, müssen sie einen Kommentar im Kommentarfeld hinzufügen. In anderen Fällen bleibt es den Nutzern überlassen, ob sie zusätzliche Details einfügen möchten.

Unten rechts auf der Abbildung \ref{Bedienungskonzept_V2} ist ein Beispiel für eine Tauschanfrage ohne besondere Merkmale zu sehen. Links neben der Übersichtsseite wird eine Tauschanfrage dargestellt, bei der ein Trainee bei der Schicht anwesend ist und die gesuchte Schicht früher stattfinden soll, mit geschriebenen Details im Kommentarfeld.

Weitere Vorschläge für zusätzliche Informationen aus der Nutzerumfrage wurden protokolliert, jedoch bislang nicht eingebunden.
\chapter{Entwicklung eines Prototypen des Benutzerkonzepts mit dem Angular-Framework}
\label{chap:prototyp}
Der Prototyp ist auf Basis des vorher entwickelten Benutzerkonzepts entwickelt. Für eine genaue Beschreibung des Benutzerkonzepts siehe Abschnitt~\ref{sec:bedienungskonzept}.

\section{Backend}
Das Backend ist mit Node.js und MongoDB entwickelt. 
Hierbei wird der Express-Server von Node.js und die Mongoose-Bibliothek von MongoDB verwendet, um eine effiziente und skalierbare Datenverwaltung zu gewährleisten.

Die Kernfunktionalitäten des Backends umfassen die Benutzerregistrierung und -verwaltung. 
In der Datei \texttt{server.js} werden die Endpunkte für die Registrierung und Anmeldung der Nutzer definiert. 
Die Datei \texttt{userModel.js} enthält das Schema für die Benutzer, welches die Felder Vorname, Nachname, E-Mail-Adresse und Passwort umfasst. 
Das Passwort wird in der \texttt{userModel.js}-Datei mithilfe der Bibliothek bcryptjs verschlüsselt, bevor es in der Datenbank gespeichert wird. 
Dies stellt sicher, dass Passwörter nicht im Klartext in der Datenbank gespeichert werden. 
 Datei \texttt{server.js} ist ebenfalls für die Überprüfung zuständig, ob ein Nutzer existiert und ob seine Anmeldedaten korrekt sind. 
 Dies ermöglicht eine sichere Authentifizierung der Nutzer.

Die Registrierung eines neuen Nutzers erfolgt über die Methode \texttt{app.post('/api/register', async (req, res) => \{ ... \})}. 
Der Nutzer gibt seine Daten ein, und bei korrekter Eingabe wird ein neues Benutzerobjekt basierend auf diesen Eingaben erstellt. 
Über die Methode \texttt{app.post('/api/login', async (req, res) => \{ ... \})} wird beim Anmeldeversuch eines Nutzer abgeglichen, ob die eingegebenen E-Mail-Adresse und das Passwort mit den in der Datenbank gespeicherten Daten übereinstimmen. 
Zunächst wird überprüft, ob ein Nutzer mit der angegebenen E-Mail-Adresse existiert. 
Falls nicht, wird eine Fehlermeldung „Benutzer nicht gefunden“ zurückgegeben. 
Anschließend wird das eingegebene Passwort mit dem in der Datenbank gehashten Passwort verglichen. 
Ist das Passwort falsch, erhält der Nutzer die Fehlermeldung „Falsches Passwort“. 
Bei korrekter Eingabe der Anmeldedaten wird die Nachricht „Login erfolgreich“ zurückgegeben.

\section{Frontend}
Das Frontend der Anwendung ist mit dem Angular Framework entwickelt. 
Alle Seiten der Anwendung sind responsiv gestaltet, sodass die Nutzer die Anwendung sowohl auf dem Handy als auch auf dem Laptop angenehm nutzen können. 
Für den Dark Mode werden derzeit die Systemeinstellungen des Geräts verwendet.

\begin{figure}[h]
    \centering
    \includegraphics[clip,width=0.8\linewidth]{images/Login_Home_dark.png}
    \caption[Login- und Übersichtsseite im Dark Mode auf einem 400x600px großen Bildschirm]{Login- und Übersichtsseite im Dark Mode auf einem 400x600px großen Bildschirm}
    \label{Login_Home_dark}
\end{figure}

Wenn ein Nutzer die Anwendung zum ersten Mal benutzt, muss er sich zunächst registrieren. 
Die Eingabefelder für die Registrierung werden validiert, um sicherzustellen, dass beispielsweise die E-Mail-Adresse im richtigen Format eingegeben wird. 
Mit der \texttt{PasswordStrengthValidator()} Funktion wird überprüft, ob das Passwort sicher genug ist. 
Da der Nutzer das Passwort zur Überprüfung zweimal eingeben muss, wird mithilfe der \texttt{passwordMatchValidator()} Funktion sichergestellt, dass beide Passwörter übereinstimmen. 
Falls dies nicht der Fall ist, wird eine Fehlermeldung angezeigt. 
Über das Augensymbol im Eingabefeld kann der Nutzer sich jedoch seine Passwörter anzeigen lassen (siehe Login-Seite in Abbildung \ref{Login_Home_dark}).

Der Service \texttt{user.service.ts} kommuniziert für Login und Registrierung über HTTP-Anfragen mit dem Express-Server im Backend. 
Nach erfolgreicher Registrierung wird der Nutzer zum Login-Formular weitergeleitet, wo er sich mit seinem erstellten Account anmelden kann. 
Wenn die Anmeldedaten korrekt sind, wird der Nutzer zur Übersichtsseite weitergeleitet, von der aus alle anderen Seiten erreichbar sind.

Die Felder, in denen ein Tauschangebot eingetragen ist, sind etwas breiter, damit die Informationen besser zu erkennen sind. 
Der heutige Tag ist farbig hervorgehoben, wie in Abbildung \ref{Login_Home_dark} zu sehen ist.

Zusätzlich kann sich der Nutzer in den Einstellungen auch wieder ausloggen. Hierfür wird die Funktion \texttt{logout()} aus der \texttt{auth.service.ts}-Datei verwendet. 
Eine Funktion zum Exportieren der Tauschangebote in den Kalender wird im Frontend bereits angezeigt, ist jedoch im Backend noch nicht implementiert.

Wenn der Nutzer auf der Übersichtsseite auf den Button mit dem Plus klickt, kann er eine neue Tauschanfrage erstellen, die Eingabefelder dafür verwenden Angular Material. 
Der Nutzer kann das Datum aus einem Kalender auswählen oder selbst eingeben. Die angebotene und gesuchte Schicht kann aus einem Dropdown-Menü ausgewählt werden. 
Wenn der Nutzer eine benutzerdefinierte Uhrzeit eingeben möchte, kann er "Benutzerdefinierte Uhrzeit" auswählen. 
Dann erscheint ein neues Eingabefeld, in dem er diese Uhrzeit eintragen kann. Durch Angular Material wird sichergestellt, dass die Uhrzeit im richtigen Format eingegeben wird. 
Eine Anfrage kann nur abgeschickt werden, wenn mindestens das Datum sowie die angebotene und gesuchte Schicht eingetragen sind.

Durch Klicken auf ein Tauschangebot im Kalender (siehe Abbildung \ref{Login_Home_dark}) öffnet sich eine Seite mit detaillierteren Informationen, je nachdem, welche Informationen für die Schicht relevant sind und vom Nutzer eingetragen wurden.
\chapter{Fazit und Ausblick}

Die manuelle Methode zur Verwaltung von Schichtwechseln über WhatsApp hat Herausforderungen und Ineffizienzen aufgezeigt. Die ständige Aktualisierung der Liste führt zu einer unübersichtlichen Informationsflut, wodurch Änderungen oder neue Einträge leicht übersehen werden können.

Durch die Anwendung des GDS konnte eine effizientere und nutzerfreundlichere Lösung zur Abstimmung von Schichtwechseln entwickelt werden. In diesem Projekt wurde der GDS an die Durchführung durch eine Einzelperson angepasst, wobei die Phasen 4 und 5 modifiziert wurden, um ein umfangreiches Bedienungskonzept zu entwickeln und anschließend daran eine Nutzerumfrage durchzuführen.

Die Nutzerumfrage, die als zentraler Bestandteil der Evaluierung des UX-Designs diente, zeigte, dass die Anwendung insgesamt positiv bewertet wurde. Die Nutzer empfanden die Funktionen als sehr intuitiv und die Übersichtlichkeit der Anwendung wurde besonders hoch bewertet. Das visuelle Design erhielt gemischte Bewertungen, wobei der Light Mode im Durchschnitt besser abschnitt als der Dark Mode. Die Mehrheit der Nutzer bevorzugte die mobile Anwendung gegenüber der Web-Anwendung. Basierend auf den Ergebnissen der Nutzerumfrage wurden einige Verbesserungsvorschläge in das Benutzerkonzept eingebaut. Darauf aufbauend wurden Prototypen mit Angular entwickelt.

Im Rahmen der Prototypenentwicklung konnten grundlegende Funktionen wie das Anlegen und Einloggen von Nutzern implementiert werden, wobei die Daten in einer MongoDB-Datenbank gespeichert werden. Der aktuelle Entwicklungsstand beschränkt sich auf die Bereitstellung dieser Funktionen im Frontend, während der Prototyp derzeit nur lokal über den Localhost verfügbar ist. Um die Anwendung in der Praxis nutzbar zu machen und einen ortsunabhängigen Zugriff zu ermöglichen, ist eine Weiterentwicklung erforderlich, bei der das Projekt beispielsweise mithilfe von Docker allgemein zugänglich gemacht wird.

Basierend auf den Ergebnissen der Nutzerumfrage und eigenen Überlegungen wurden verschiedene Verbesserungsvorschläge erarbeitet, um die Nutzerfreundlichkeit und Funktionalität der Anwendung weiter zu optimieren. Dazu gehört die Implementierung von Benachrichtigungen, die Nutzer darüber informieren, wenn ihre erstellte Schicht getauscht wurde oder ein neues Tauschangebot vorliegt. Zudem soll die Möglichkeit bestehen, diese Benachrichtigungen in den Einstellungen zu deaktivieren. 

Eine Integration des Dienstplans soll sicherstellen, dass nur diejenigen Nutzer ein Tauschangebot annehmen können, die tatsächlich in der gesuchten Schicht eingetragen sind. Bisher wird darauf vertraut, dass die Nutzer diese Regel eigenständig einhalten. Mit dieser Funktion könnten alle potenziellen Tauschpartner angezeigt werden, was die Übersicht und Organisation erheblich erleichtert. Außerdem wäre es für die Nutzer von Vorteil, eine Ringtausch-Funktion zu implementieren, die den Austausch von Schichten zwischen mehreren Personen vereinfacht.

Ein weiterer Aspekt betrifft die Anpassung der Farbmodi. Bisher orientieren sich die Farbmodi der Anwendung an den Systempräferenzen der Nutzer. Dafür kann eine Funktion in den Einstellungen eingebaut werden, die es den Nutzern ermöglicht, direkt zwischen einem Dark Mode und einem Light Mode zu wählen.

\backmatter
\sloppy
\pagenumbering{Roman}
\printbibliography
\addcontentsline{toc}{chapter}{Literaturverzeichnis}
\pagenumbering{Roman}


%%%%%% Anhang %%%%%%
\appendix
\chapter*{Anhang}
\addcontentsline{toc}{chapter}{Anhang}

\section*{Fragebogen der Nutzerumfrage zur Bewertung der UX}
\label{app:fragebogenNutzerumfrage}

\noindent
\includegraphics[width=\textwidth]{appendix/Frage1-3.png}
\par\vspace{\baselineskip} 

\noindent
\includegraphics[width=\textwidth]{appendix/Frage4-7.png}
\par\vspace{\baselineskip} 

\noindent
\includegraphics[width=\textwidth]{appendix/Frage8-11.png}

\end{document}