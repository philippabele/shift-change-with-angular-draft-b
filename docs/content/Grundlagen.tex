\chapter{Grundlagen}

\section{Google Design Sprint}
Der Google Design Sprint \ac{GDS} wurde konzipiert, um Organisationen bei der Bewältigung von Herausforderungen und Problemen zu unterstützen. Insbesondere die, die durch die dynamische Natur des Marktes und sich verändernde Produktanforderungen entstehen. Im Rahmen dieses Konzepts ist es das Ziel, innerhalb eines Zeitraums von fünf Tagen einen Prototyp zu erstellen und zu evaluieren \cite[vgl.][S.98]{Design_Sprint}.
Die fünftägige Methode:
Diese Methode bietet den Vorteil, dass nicht auf die Markteinführung gewartet werden muss, um Feedback zu erhalten. Stattdessen können dringende Fragen sofort beantwortet werden.
Am ersten Tag wird die Grundlage und der Schwerpunkt für die Woche festgelegt. Am zweiten Tag konzentriert sich das Team auf die Bewältigung bereits bekannter Herausforderungen. Anders als bei herkömmlichen Brainstorming-Sitzungen arbeiten die Teammitglieder eigenständig an Lösungsansätzen und folgen einem strukturierten vierstufigen Prozess, um das kritische Denken zu fördern.
Am dritten Tag muss das Team aus den verschiedenen Lösungsansätzen wählen, welche als Prototyp entwickelt und getestet werden sollen. Dabei kommt die fünfstufige "Sticky Decision"-Methode zum Einsatz, um die besten Lösungen zu identifizieren. Anschließend wird ein detaillierter Prozessplan für den Prototypen erstellt. 
Der vierte Tag ist der Erstellung eines realistischen Prototyps gewidmet, der dann dem Kunden präsentiert werden kann. Am letzten Tag werden fünf Kunden in Einzelgesprächen der Prototyp vorgestellt, um ihr Feedback zu erhalten \cite[vgl.][S.99]{Design_Sprint}.

\section{Google Design Sprint in diesem Projekt}