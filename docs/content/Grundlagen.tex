\chapter{Grundlagen}

\section{Google Design Sprint}

Der Google Design Sprint \ac{GDS} ist eine von Jake Knapp bei Google Ventures entwickelte Methodik zur schnellen und effizienten Problemlösung sowie Produktentwicklung, insbesondere für Herausforderungen, die sich aus der dynamischen Natur des Marktes und den sich verändernden Produktanforderungen ergeben. 
Ziel dieser Methode ist es, innerhalb eines Zeitraums von fünf Tagen einen Prototyp zu entwickeln und zu evaluieren. 
Diese Methode bietet den Vorteil, dass nicht auf die Markteinführung gewartet werden muss, um Feedback zu erhalten. Stattdessen können dringende Fragen sofort beantwortet werden \cite[S.98 f.]{Design_Sprint}.\\
Die fünftägige Methode, wie in Abbildung \ref{fig:GDS} dargestellt, verläuft wie folgt:\\

\begin{figure}[h]
	\centering
	\includegraphics[width=0.75\textwidth]{images/GDS.png}
	\caption[Ablauf eines GDS]{Ablauf eines GDS \cite{GDS_Abbildung}}
	\label{fig:GDS}
\end{figure}

Am ersten Tag geht es darum, das Problem zu verstehen und den Fokus für die Woche festzulegen. Das Team definiert das langfristige Ziel und identifiziert die Herausforderung. \\
Am zweiten Tag konzentriert sich das Team auf die Bewältigung bereits bekannter Herausforderungen. Anders als bei herkömmlichen Brainstorming-Sitzungen arbeiten die Teammitglieder einzeln an Lösungsansätzen und folgen einem strukturierten vierstufigen Prozess, um das kritische Denken zu fördern. \\
Am dritten Tag trifft das Team Entscheidungen darüber, welche Idee als Prototyp entwickelt und getestet werden sollen. Dabei kommt die fünfstufige "Sticky Decision"-Methode zum Einsatz, um die besten Lösungen zu identifizieren. Anschließend wird ein detaillierter Prozessplan für den Prototypen erstellt. \\
Am vierten Tag wird ein realitätsnaher Prototyp entwickelt. Das Ziel ist es, eine testbare Version der Lösung zu erstellen, die am nächsten Tag mit echten Nutzern evaluiert werden kann. \\
Der letzte Tag ist für das Testen des Prototyps reserviert. Das Team sammelt Feedback von echten Nutzern und erhält wertvolle Einsichten, ob die Lösung in der Praxis funktioniert und welche Anpassungen nötig sind. Dieses Feedback ist entscheidend, um die Stärken und Schwächen der entwickelten Lösung zu identifizieren \cite[S.22 ff.]{Design_Sprint}.

\section{User Experience}


\section{Technologien}
\subsection{Angular}
Angular ist eine Plattform zur Entwicklung von Webanwendungen, die auf TypeScript \ac{TS} basiert. 
Mit Angular können Single Page Applications für Web-, Mobil- und Desktop-Anwendungen entwickelt werden. 
Es ist Open Source und wird von Google unterstützt, was bedeutet, dass es kostenlos verwendet werden kann und von einer großen Entwickler-Community gepflegt wird. 
Ursprünglich war es als AngularJS bekannt und nutzte JavaScript als Hauptprogrammiersprache. Die Codebasis wurde jedoch komplett neu geschrieben, wobei JavaScript durch TS ersetzt wurde \cite{angular_arch}.\\

\begin{figure}[h]
	\centering
	\includegraphics[width=0.5\textwidth]{images/Komponente.png}
	\caption[Komponenten in Angular]{Komponenten in Angular}
	\label{fig:KomponentenAngular}
\end{figure}
Die grundlegenden Bausteine des Angular-Frameworks sind die Komponenten, welche eine isolierte und somit wieder verwendbar Einheit sind. 
Eine Komponente besteht aus einer HTML- und CSS-Datei (siehe Abbildung \ref{fig:KomponentenAngular}), die das Aussehen der Benutzeroberfläche definieren. Die zugehörige Klasse enthält den TS Code, der das Verhalten der Komponente steuert \cite{angular_arch}.\\

\begin{figure}[h]
	\centering
	\includegraphics[width=0.5\textwidth]{images/@Component.png}
	\caption[@Component-Dekorator in TS]{@Component-Dekorator in TS}
	\label{fig:@Component}
\end{figure}
Der @Component-Dekorator in TS fügt der Klasse zusätzliche Metadaten hinzu, wie den Pfad zum Template und zu den Styles (siehe templateUrl und styleUrls in der Abbildung \ref{fig:@Component}). Die .spec.ts Datei wird für die Unit-Tests verwendet.
Der Routing-Service ermöglicht die Navigation innerhalb der Anwendung und zeigt verschiedene Komponenten basierend auf der URL an \cite{angular_arch}.\\

\subsection{}