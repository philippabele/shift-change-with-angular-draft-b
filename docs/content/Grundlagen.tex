\chapter{Grundlagen}

\section{Google Design Sprint}

Der \ac{GDS} ist eine von Jake Knapp bei Google Ventures entwickelte Methodik zur schnellen und effizienten Problemlösung sowie Produktentwicklung, insbesondere für Herausforderungen, die sich aus der dynamischen Natur des Marktes und den sich verändernden Produktanforderungen ergeben. 
Ziel dieser Methode ist es, innerhalb eines Zeitraums von fünf Tagen einen Prototyp zu entwickeln und zu evaluieren. 
Diese Methode bietet den Vorteil, dass nicht auf die Markteinführung gewartet werden muss, um Feedback zu erhalten. Stattdessen können dringende Fragen sofort beantwortet werden \cite[S.98 f.]{Design_Sprint}.

Die fünftägige Methode, wie in Abbildung \ref{GDS} dargestellt, verläuft wie folgt:

\begin{figure}[h]
    \centering
    \includegraphics[clip,width=0.75\linewidth]{images/GDS.png}
    \caption[Ablauf eines GDS]{Ablauf eines GDS \cite{GDS_Abbildung}}
    \label{GDS}
\end{figure}

Am ersten Tag geht es darum, das Problem zu verstehen und den Fokus für die Woche festzulegen. Das Team definiert das langfristige Ziel und identifiziert die Herausforderung. 

Am zweiten Tag konzentriert sich das Team auf die Bewältigung bereits bekannter Herausforderungen. Anders als bei herkömmlichen Brainstorming-Sitzungen arbeiten die Teammitglieder einzeln an Lösungsansätzen und folgen einem strukturierten vierstufigen Prozess, um das kritische Denken zu fördern. 

Am dritten Tag trifft das Team Entscheidungen darüber, welche Idee als Prototyp entwickelt und getestet werden sollen. Dabei kommt die fünfstufige „Sticky Decision“-Methode zum Einsatz, um die besten Lösungen zu identifizieren. Anschließend wird ein detaillierter Prozessplan für den Prototypen erstellt. 

Am vierten Tag wird ein realitätsnaher Prototyp entwickelt. Das Ziel ist es, eine testbare Version der Lösung zu erstellen, die am nächsten Tag mit echten Nutzern evaluiert werden kann. 

Der letzte Tag ist für das Testen des Prototyps reserviert. Das Team sammelt Feedback von echten Nutzern und erhält Einsicht, ob die Lösung in der Praxis funktioniert und welche Anpassungen nötig sind. Dieses Feedback ist entscheidend, um die Stärken und Schwächen der entwickelten Lösung zu identifizieren \cite[S.22 ff.]{Design_Sprint}.

\section{User Experience}
Das Design der \ac{UX} konzentriert sich auf jedes Element und jede Funktion, die der Benutzer bei einer Anwendung sieht, mit dem Ziel, eine möglichst angenehme und effiziente Erfahrung zu ermöglichen \cite[S.12]{Bordegoni}. 
Laut der International Organization for Standardization (ISO 9241-210) wird UX definiert als „Wahrnehmungen und Reaktionen einer Person, die sich aus der Verwendung und/oder der erwarteten Verwendung eines Produkts, Systems oder einer Dienstleistung ergeben” \cite{iso}. 
Während sich das User Interface auf das Erscheinungsbild der Anwendung konzentriert, beispielsweise auf Schriftarten und Farben, geht die UX tiefer und umfasst das gesamte Erlebnis des Nutzers \cite[S.8]{Canziba}.

In den letzten Jahrzehnten hat die Bedeutung der User Experience stark zugenommen, da Unternehmen erkannt haben, dass ein positives Nutzungserlebnis entscheidend für den Erfolg eines Produkts ist \cite{ux_article}. 
Bei einem schlechten UX-Design fällt es den Benutzern schwer, die Anwendung zu nutzen, und sie wechseln, sobald sie eine ähnliche, bessere App gefunden haben, die die gleiche Aufgabe erfüllt. 
Deshalb ist es für ein gutes UX-Design wichtig, dass der Designer die Bedürfnisse der Benutzer vorhersieht und erfüllt, indem er ihre Sichtweise einnimmt. Ein gutes UX-Design kann die Produktivität steigern, die Zufriedenheit der Kunden erhöhen und die Verkaufszahlen verbessern. 
Zudem können durch eine optimierte Benutzererfahrung die Kosten für Support und Wartung gesenkt werden \cite[S.8 ff.]{Canziba}.

Die UX umfasst verschiedene Komponenten, die zusammenarbeiten, um ein positives Nutzungserlebnis zu schaffen. Dazu gehören Usability, Ästhetik, Interaktion und Emotionalität.

Die Usability bezieht sich auf die Benutzerfreundlichkeit und Effizienz eines Produkts. 
Ein System mit hoher Usability ermöglicht es den Nutzern, ihre Ziele schnell und ohne Frustration zu erreichen. 
Dies beinhaltet Aspekte wie Lernfähigkeit, Effizienz der Nutzung und Zufriedenheit der Nutzer \cite[S.23 ff.]{Nielsen}.

Die Ästhetik bezieht sich auf das Erscheinungsbild des Produkts. 
Ein ansprechendes visuelles Design kann die Wahrnehmung der Qualität und den Gesamteindruck eines Produkts erheblich verbessern. 
Studien zeigen, dass ästhetisch ansprechende Designs oft als funktionaler und benutzerfreundlicher wahrgenommen werden \cite{Tractinsky}.

Die Interaktion beschreibt, wie Nutzer mit einem Produkt oder einer Dienstleistung umgehen. 
Dies umfasst die Art und Weise, wie Eingaben getätigt werden, wie das System auf Nutzeraktionen reagiert und wie intuitiv diese Interaktionen gestaltet sind. 
Gute Interaktionsdesigns minimieren die kognitive Belastung und fördern eine reibungslose und intuitive Nutzung \cite{interaction_design}.

Emotionalität in der UX bezieht sich auf die emotionalen Reaktionen und das subjektive Erleben der Nutzer während der Interaktion mit einem Produkt. 
Positive emotionale Erfahrungen können die Benutzerbindung und Zufriedenheit erhöhen. 
Studien belegen, dass Produkte, die positive Emotionen hervorrufen, häufiger genutzt und weiterempfohlen werden \cite{emotion}.

Es gibt eine Vielzahl von Methoden zur Bewertung der UX, zu den gebräuchlichsten Methoden gehören:

Usability-Tests beinhalten die Beobachtung und Analyse der Nutzerinteraktionen mit einem System, um Nutzerprobleme zu identifizieren. 
Durch das Testen in realistischen Nutzungsszenarien können konkrete Probleme und deren Schweregrad ermittelt werden, was zur Verbesserung der Nutzerfreundlichkeit beiträgt. 
Diese Methode ist besonders effektiv, um zu verstehen, wie Nutzer tatsächlich mit dem System interagieren \cite[S.165 ff.]{Nielsen}.

Mit Nutzerbefragungen und Interviews können qualitative Daten direkt von den Nutzern gesammelt werden. 
Diese Methode ermöglicht es, tiefergehende Einblicke in die Meinungen, Erfahrungen und Verbesserungsvorschläge der Nutzer zu gewinnen, was zur Optimierung der Interaktionen und der allgemeinen Zufriedenheit beitragen kann. 
Durch gezielte Fragen können spezifische Aspekte der Nutzererfahrung identifiziert und adressiert werden \cite{ux_bewertung_interview}.

Bei der Eye-Tracking-Technologie wird die Augenbewegungen von Nutzern verfolgt, wie in Abbildung \ref{eye_tracking} dargestellt. 

\begin{figure}[h]
    \centering
    \includegraphics[clip,width=0.4\linewidth]{images/Eye-Tracking-Google-Search-Heat-Map.jpg}
    \caption[Beispielhafte Visualisierung der Eye Tracking Methode anhand einer Heat Map]{Beispielhafte Visualisierung der Eye Tracking Methode anhand einer Heat Map \cite{image_eye_tracking}}
    \label{eye_tracking}
\end{figure}

Durch die Messung von Fixationen, also den Momenten, in denen das Auge auf einem Punkt verharrt, kann festgestellt werden, welche Elemente einer Benutzeroberfläche besondere Aufmerksamkeit erregen. 
Sakkaden, die schnellen Augenbewegungen zwischen Fixationen, helfen dabei zu verstehen, wie Nutzer Informationen auf einer Seite scannen. 
Diese Daten liefern wertvolle Einblicke in die Nutzerinteraktion und ermöglichen die Optimierung von Design und Layout basierend auf tatsächlichen Nutzerverhalten \cite[S.3 ff.]{eye_tracking}.
  
Durch die Kombination dieser Methoden können umfassende Einblicke in die Stärken und Schwächen der User Experience gewonnen werden, die zur kontinuierlichen Verbesserung des Produkts beitragen.

\section{Technologien}

\subsection{Das Angular-Framework}
Angular ist eine Plattform zur Entwicklung von Webanwendungen, die auf \ac{TS} basiert. 
Mit Angular können Single Page Applications für Web-, Mobil- und Desktop-Anwendungen entwickelt werden. 
Es ist Open Source und wird von Google unterstützt, was bedeutet, dass es kostenlos verwendet werden kann und von einer großen Entwickler-Community gepflegt wird. 
Ursprünglich war es als AngularJS bekannt und nutzte JavaScript als Hauptprogrammiersprache. Die Codebasis wurde jedoch komplett neu geschrieben, wobei JavaScript durch TS ersetzt wurde \cite{angular_arch}.

\begin{figure}[h]
    \centering
    \includegraphics[clip,width=0.5\linewidth]{images/Komponente.png}
    \caption[Komponenten in Angular]{Komponenten in Angular}
    \label{KomponentenAngular}
\end{figure}

Die grundlegenden Bausteine des Angular-Frameworks sind die Komponenten, welche eine isolierte und somit wieder verwendbar Einheit sind. 
Eine Komponente besteht aus einer HTML- und CSS-Datei (siehe Abbildung \ref{KomponentenAngular}), die das Aussehen der Benutzeroberfläche definieren. Die zugehörige Klasse enthält den TS Code, der das Verhalten der Komponente steuert \cite{angular_arch}.

\begin{figure}[h]
    \centering
    \includegraphics[clip,width=0.5\linewidth]{images/@Component.png}
    \caption[@Component-Dekorator in TypeScript]{@Component-Dekorator in TS}
    \label{Component}
\end{figure}

Der @Component-Dekorator in TS fügt der Klasse zusätzliche Metadaten hinzu, wie den Pfad zum Template und zu den Styles (siehe templateUrl und styleUrls in der Abbildung \ref{Component}) \cite{angular_arch}.

Durch Unit-Tests in Angular ist es möglich, isoliert einzelne Teile der Anwendung zu testen, um sicherzustellen, dass diese gemäß den Erwartungen funktionieren. 
Diese Tests sind in .spec.ts Dateien organisiert und umfassen Testfälle für Komponenten, Dienste und Module. 
Sie werden üblicherweise mit Testing-Frameworks wie Jasmine und Karma ausgeführt, die in die Angular-Entwicklungsumgebung integriert sind. 
Neben Unit-Tests gibt es auch End-to-End-Tests, die das Verhalten der Anwendung in einer realen Umgebung simulieren \cite[S.340 ff.]{angular_testing}.

Der Routing-Service ermöglicht es verschiedene Ansichten basierend auf der URL der Anwendung zu laden. 
Durch die Definition von Routen und deren Zuordnung zu spezifischen Komponenten ermöglicht Angular eine klare Strukturierung und Navigation innerhalb der Anwendung, was insbesondere für komplexe Single Page Applications entscheidend ist. 
Routen können parameterisiert werden oder der Zugriff auf bestimmte Routen kann durch Guards gesteuert werden \cite{angular_routes}.

\subsection{MongoDB}
MongoDB ist eine dokumentenorientierte NoSQL-Datenbank und entstand 2007 aus einem Platform-as-a-Service-Projekt von dem Start Up 10gen, das auf automatisiertes Infrastrukturmanagement abzielte. 
Aufgrund der Ablehnung durch Entwickler, die Kontrolle über ihre Technologie-Stacks zu verlieren, fokussierte sich 10gen auf die Datenbank und benannte sich in MongoDB, Inc. um. 
MongoDB wird als Open-Source-Projekt entwickelt, wobei die Community zur Mitarbeit ermutigt wird, das Geschäftsmodell umfasst die Unterstützung der Open-Source-Entwicklung und das Anbieten von Abonnementdiensten \cite[S.5 f.]{Banker2016MongoDB}.

MongoDB wurde zur schnellen Entwicklung von Webanwendungen und Internet-Infrastruktur designed. 
Es ist möglich, ein strukturiertes Dokument in der Datenbank zu speichern, ohne bei der Erstellung der Datenbank schon wissen zu müssen, wie die Datenstruktur aussehen wird oder neue Tabellen erstellen zu müssen, wenn sich die Daten ändern. 
Die Struktur wird durch den Code der Anwendung bestimmt. Außerdem ist es, wenn für die Speicherung der Datenmenge mehr als eine Datenbank benötigt wird, mit MongoDB deutlich einfacher und performante diese zu skalieren, als bei einer relationalen Datenbank \cite[S.4 ff.]{Banker2016MongoDB}. 

Intern speichert MongoDB die Daten als Binary \ac{JSON}, eine Struktur, die JSON ähnelt, jedoch für die Speicherung vieler Dokumente optimiert ist. 
Nach einer Abfrage werden die Ergebnisse in eine leicht lesbare Datenstruktur übersetzt, das Dokumentenformat JSON, das aus Schlüsseln und Werten besteht (für Beispiel siehe Abbildung \ref{MongoDB_Document}). 
Während relationale Datenbanken Tabellen verwenden, nutzt MongoDB stattdessen Kollektionen \cite[S.7]{Banker2016MongoDB}.

\begin{figure}[h]
    \centering
    \includegraphics[clip,width=0.5\linewidth]{images/MongoDB_Document.png}
    \caption[Beispiel eines Dokuments in einer MongoDB]{Beispiel eines Dokuments in einer MongoDB}
    \label{MongoDB_Document}
\end{figure}

Insgesamt bietet MongoDB eine leistungsfähige, flexible und skalierbare Lösung für die Verwaltung komplexer und dynamischer Datenmodelle, was sie ideal für moderne Webanwendungen macht.

\subsection{Node.js}
Node.js ist eine serverseitige Laufzeitumgebung, die auf der V8-JavaScript-Engine von Google Chrome basiert und die Ausführung von JavaScript-Code außerhalb des Browsers ermöglicht. 
Entwickelt von Ryan Dahl im Jahr 2009, hat sich Node.js schnell als leistungsfähige Plattform für die Erstellung von serverseitigen Anwendungen etabliert \cite[S.3 ff.]{Cantelon2014}. 
Ein wesentliches Merkmal von Node.js ist sein ereignisgesteuertes, nicht-blockierendes I/O-Modell, das eine effiziente Verarbeitung gleichzeitiger Verbindungen ermöglicht \cite{Tilkov2010}. 
Diese Architektur ist besonders für Anwendungen geeignet, die häufige I/O-Operationen ausführen, wie beispielsweise APIs oder kleine Webserver. 
Ein weiterer Vorteil von Node.js ist die Nutzung von JavaScript sowohl auf der Client- als auch auf der Serverseite, was zu einer einheitlichen Codebasis und vereinfachten Entwicklungsprozessen führt . 
Die Kombination von Angular und Node.js bietet daher eine schnelle und effiziente Datenverarbeitung sowie eine einheitliche Entwicklungsumgebung \cite[S.5 ff.]{Casciaro2020node}.