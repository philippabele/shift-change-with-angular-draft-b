https://link.springer.com/chapter/10.1007/978-3-030-29384-0_6

Jake Knapp stellte fest:

- während großen Brainstorming Meetings, die besten Ideen oft von Einzelpersonen entwickelt wurden, die eine große Herausforderung und nicht allzu viel Zeit hatten, um daran zu arbeiten
- wichtiger Faktor: alle an einem Projekt beteiligten Personen gemeinsam in einem Raum arbeiteten, um ihren eigenen Teil des Problems zu lösen, und bereit waren, Fragen zu beantworten

→ Kombination von individueller Arbeit, Zeit für Prototypen und einer unausweichlichen Frist nannte Knapp diese konzentrierten Designbemühungen "Sprints”

Google Design Sprint (GDS)

- Verfahren zur Lösung von Problemen und zur Erprobung neuer Ideen
- innerhalb von fünf Tagen (Entwurf, Skizze, Entscheidung, Prototyp, Test) ein Prototyp bauen und testen
- Reaktion der Kunden zu beobachten, bevor man sich zum Bau eines echten Produkts verpflichtet

\cite[vgl.][]{GDS}

die 5 Tage:

- Montag: Grundlage und Schwerpunkt für die Woche schaffen
- Dienstag:
    - Probleme lösen
    - kein typisches Gruppen-Brainstorming
    - jeder Einzeln Lösungen skizzieren - kritisches Denken
- Mittwoch:
    - verschiedene Lösungen, im Team entscheiden welche als Prototyp entwickelt und getestet werden sollen
    - fünfstufige "Sticky Decision"-Methode anwenden, um die besten Lösungen zu ermitteln → entscheiden
    - Storyboard erstellen: ein Schritt-für-Schritt-Plan für Ihren Prototyp
- Donnerstag: realistischen Prototyp bauen, um Kunden vorstellen zu können
- Freitag:
    - Prototyp fünf Kunden in fünf separaten 1:1-Gesprächen vorstellen
    → Anstatt auf Markteinführung zu warten, um perfekte Daten zu erhalten, erhalten von sofortigen Antworten auf dringendste Fragen