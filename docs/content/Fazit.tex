\chapter{Fazit und Ausblick}

Die manuelle Methode zur Verwaltung von Schichtwechseln über WhatsApp hat Herausforderungen und Ineffizienzen aufgezeigt. Die ständige Aktualisierung der Liste führt zu einer unübersichtlichen Informationsflut, wodurch Änderungen oder neue Einträge leicht übersehen werden können.

Durch die Anwendung des GDS konnte eine effizientere und nutzerfreundlichere Lösung zur Abstimmung von Schichtwechseln entwickelt werden. In diesem Projekt wurde der GDS an die Durchführung durch eine Einzelperson angepasst, wobei die Phasen 4 und 5 modifiziert wurden, um ein umfangreiches Bedienungskonzept zu entwickeln und anschließend daran eine Nutzerumfrage durchzuführen.

Die Nutzerumfrage, die als zentraler Bestandteil der Evaluierung des UX-Designs diente, zeigte, dass die Anwendung insgesamt positiv bewertet wurde. Die Nutzer empfanden die Funktionen als sehr intuitiv und die Übersichtlichkeit der Anwendung wurde besonders hoch bewertet. Das visuelle Design erhielt positive, aber gemischte Bewertungen, dabei schnitt der Light Mode im Durchschnitt besser ab als der Dark Mode. Die Mehrheit der Nutzer bevorzugte die mobile Anwendung gegenüber der Web-Anwendung. Basierend auf den Ergebnissen der Nutzerumfrage wurden einige Verbesserungsvorschläge in das Benutzerkonzept eingebaut. Darauf aufbauend wurden Prototypen mit Angular entwickelt.

Im Rahmen der Prototypenentwicklung konnten grundlegende Funktionen wie das Anlegen und Einloggen von Nutzern implementiert werden, wobei die Daten in einer MongoDB-Datenbank gespeichert werden. Der aktuelle Entwicklungsstand beschränkt sich auf die Bereitstellung dieser Funktionen im Frontend, während der Prototyp derzeit nur lokal über den Localhost verfügbar ist. Um die Anwendung in der Praxis nutzbar zu machen und einen ortsunabhängigen Zugriff zu ermöglichen, ist eine Weiterentwicklung erforderlich, bei der das Projekt beispielsweise mithilfe von Docker allgemein zugänglich gemacht wird.

Basierend auf den Ergebnissen der Nutzerumfrage und eigenen Überlegungen wurden verschiedene Verbesserungsvorschläge erarbeitet, um die Nutzerfreundlichkeit und Funktionalität der Anwendung weiter zu optimieren. Dazu gehört die Implementierung von Benachrichtigungen, die Nutzer darüber informieren, wenn ihre erstellte Schicht getauscht wurde oder ein neues Tauschangebot vorliegt. Zudem soll die Möglichkeit bestehen, diese Benachrichtigungen in den Einstellungen zu deaktivieren. 

Eine Integration des Dienstplans soll sicherstellen, dass nur diejenigen Nutzer ein Tauschangebot annehmen können, die tatsächlich in der gesuchten Schicht eingetragen sind. Bisher wird darauf vertraut, dass die Nutzer diese Regel eigenständig einhalten. Mit dieser Funktion könnten alle potenziellen Tauschpartner angezeigt werden, was die Übersicht und Organisation erheblich erleichtert. Außerdem wäre es für die Nutzer von Vorteil, eine Ringtausch-Funktion zu implementieren, die den Austausch von Schichten zwischen mehreren Personen vereinfacht.

Ein weiterer Aspekt betrifft die Anpassung der Farbmodi. Bisher orientieren sich die Farbmodi der Anwendung an den Systempräferenzen der Nutzer. Dafür kann eine Funktion in den Einstellungen eingebaut werden, die es den Nutzern ermöglicht, direkt zwischen einem Dark Mode und einem Light Mode zu wählen.